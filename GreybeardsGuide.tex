\documentclass[t]{beamer}

\usepackage[english]{babel}

\usepackage{float}
%\floatstyle{boxed}
%\restylefloat{figure}
\usepackage{times}
%\usepackage[T1]{fontenc}
\usepackage{graphicx}
\usepackage{hyperref}
\usepackage{listings}
\usepackage[utf8]{inputenc}
\usepackage{graphicx}
%% \usepackage{pgf,tikz}
%% \usepackage{pgfplots}
%% \usetikzlibrary{shapes,arrows,snakes,automata,backgrounds,petri,calc,trees}
%% \usetikzlibrary{matrix, positioning, fit}
%% \usepackage{smartdiagram}
%% \usepackage{mathtools}
\usepackage{ulem}


%\setbeameroption{show only notes}

\def\PresTitle{Greybeard's Guide to *NIX}
\font\footnoteFont=phvr7t at 8pt
\font\footnoteRefFont=phvro7t at 8pt
\font\thFont=phvb7t at 12pt
\newfont{\codeFont}{cmtt10 scaled 800}
\setbeamerfont{verbatim}{size={\fontsize{8pt}{10pt}}}
\mode<presentation>
{
  \usetheme{Madrid}
  \setbeamercovered{transparent}
  \usecolortheme{orchid}
}


\title[\PresTitle]{\PresTitle}

\author[Steve Roggenkamp] % (optional, use only with lots of authors)
{Steve Roggenkamp\\
}

\date[13 Sep 2017] % (optional)
{13 Sep 2017 \\
 }

\institute[]{
 Institute for Biomedical Informatics\\
 
 University of Kentucky\\
 
 \href{mailto:steve.roggenkamp@uky.edu}{\nolinkurl{steve.roggenkamp@uky.edu}}
}

\subject{\PresTitle}

\begin{document}

\frame{\titlepage}

% \begin{frame}{}
%  \begin{itemize}
%  \item Why learn to use the command line?
%  \item Six things to know
%  \end{itemize}
% \end{frame}

\begin{frame}{Why learn to use the command line?}
  \begin{itemize}
  \item Systems administration
  \item Resource constrained systems
    \begin{itemize}
    \item Embedded systems
    \item High performance systems
    \end{itemize}
  \item It can do a lot for you
  \end{itemize}
  \note{}
\end{frame}

\begin{frame}{The Shell}
  \begin{itemize}
    \item Parses your input and executes commands
    \item Provides an environment in which you work
    \item Provides a scripting programming language
  \end{itemize}
\end{frame}

\begin{frame}{Shells - If you don't like one, try another}
  \begin{itemize}
  \item \texttt{ash}, \texttt{bash}, \texttt{dash}
  \item Bourne shell, Korn shell, PD Korn,  Bourne-again shell
  \item \texttt{csh}, \texttt{tcsh}, \texttt{zsh}
  \item Busybox (which normally uses \texttt{ash})
  \item Each provides unique set of capabilities
  \item Today's talk focuses on Bourne-again shell, aka, \texttt{bash}
    \begin{itemize}
    \item Default shell for most Linux and Mac OSX
      \item Available for most *NIX system
    \end{itemize}
  \end{itemize}
  \note{}
\end{frame}

\begin{frame}{Starting a shell}
  \begin{itemize}
  \item Start a terminal in a GUI environment
  \item Remotely execute a shell via \texttt{ssh}
  \item Log into a system console
  \end{itemize}
  \note{}
\end{frame}

\begin{frame}{Sample session start}
  \includegraphics[width=10cm,scale=0.4]{images/newtty-1.png}

  Now what?? Coder's Block!!!
  \note{}
\end{frame}

\begin{frame}{Session - list current directory}
  \includegraphics[width=10cm,scale=0.4]{images/newtty-2.png}

  \texttt{ls} lists the contents of a directory
  \note{}
\end{frame}

\begin{frame}{Shell - file redirection}
%  \includegraphics[width=10cm,scale=0.4]{images/man-k.png}
  \begin{itemize}
  \item 
  \end{itemize}
  \note{
%%        echo Hello world! > hello.txt
%%        echo ``A second line'' >> hello.txt
%%        ls -l >ls.out
%%        cat < hello.txt
%%        cat *
%%        file *
%%        echo ''ls -l'' >>cmd.txt
%%        . cmd.txt
%%        echo ``#!/bin/bash'' >cmd1.txt
%%        echo ''ls -al'' >>cmd1.txt
%%        cat >cmd2.txt <<EOF
%%        #!/bin/bash
%%        ls -al *.txt
%%        EOF
       }
\end{frame}

\begin{frame}{Shell - executing scripts}
%  \includegraphics[width=10cm,scale=0.4]{images/man-k.png}
  \begin{itemize}
  \item 
  \end{itemize}
  \note{
%%        . cmd.txt
%%        source cmd.txt
%%        chmod +x cmd[12].txt
%%        ./cmd1.txt
%%        file *
%%        ./cmd2.txt
}
\end{frame}

\begin{frame}{Shell - filename patterns }
%%  \includegraphics[width=10cm,scale=0.4]{images/man-k.png}
  \begin{itemize}
  \item 'file.out' matches a file named 'file.out'
  \item *NIX does not respect ``filename extensions''
  \item * matches any numbers of characters
  \item ? matches a single character
  \item \[xy\] matches characters 'x' and 'y'
  \item \[x-y\] matches characters between 'x' and 'y' inclusive
  \end{itemize}
  \note{}
\end{frame}



\begin{frame}{What to type for a given command}
%%  \includegraphics[width=10cm,scale=0.4]{images/newtty-3.png}

  \begin{itemize}
  \item \texttt{man ls} provides the manual page for \texttt{ls}
  \end{itemize}
  \note{}
\end{frame}

\begin{frame}{Man pages}
  \begin{itemize}
  \item \texttt{man} pages contain multiple sections
  \item Typical sections include:
    \begin{itemize}
    \item NAME - name of the command and brief discussion
    \item SYNOPSIS - how to invoke the command and list of arguments
    \item DESCRIPTION - detailed description of the program
    \item OPTIONS - options and their description
    \item EXAMPLES - example invocations and what they do
    \item EXIT STATUS - \texttt{exit(2)} codes (0 indicates success) 
    \item ENVIROMENT - how shell variables affect program
    \item SEE ALSO - other programs or documents to consult
    \item BUGS - known issues
    \end{itemize}
  \end{itemize}
  \note{}
\end{frame}

\begin{frame}{Man pages - cont.}
  \begin{itemize}
  \item UNIX documentation system
  \item Organized into multiple sections
    \begin{itemize}
    \item 1 - programs or shell commands
    \item 2 - System (kernel) calls
    \item 3 - Library calls
    \item 4 - Special files (usually found in /dev)
    \item 5 - File formats
    \item 6 - Games
    \item 7 - Miscellaneous
    \item 8 - System administration commands (root)
    \item 9 - Non-standard kernel commands
    \end{itemize}
  \end{itemize}
  \note{}
\end{frame}

\begin{frame}{Man pages - How to find commands}
%%  \includegraphics[width=10cm,scale=0.4]{images/man-k.png}
  \begin{itemize}
  \item 
  \end{itemize}
  \note{man -k 'add.*user'}
\end{frame}

\begin{frame}{Pipes - Composing programs}
  \begin{itemize}
  \item Pipes feed the output from a program to the input of another program
  \item Allow us to string programs together for ``one-of'' programs
  \item Very efficient 
    \begin{itemize}
    \item Creates one process per program
    \item File buffer between programs
    \item Signals and scheduler coordinate writing and reading data
      between programs - no extra files needed
    \end{itemize}
  \end{itemize}
  \note{}
\end{frame}

\begin{frame}{Regular expressions - searching strings and finding things}
  \begin{itemize}
  \item Regular expressions allow you to search using string patterns
` \item Many programs in *NIX make use of regular expressions:
  \item Most characters match themselves, thus 'abc' matches the
    string  ``abc''
  \item Meta characters:
    \begin{itemize}
    \item \^{ }  matches the beginning of a line
    \item \$ matches the end of a line
    \item . matches a single character
    \item ? matches zero or one of the preceding pattern
    \item * matches zero or more of the preceding pattern
    \item + matches zero or more of the preceding pattern
    \item $\lbrack$ and $\rbrack$ indicate the start and end of a character class
    \item ( and ) indicate the start and end of an atom
    \item $\mid$ indicates an alternate
    \item $\setminus$ before one of the above indicates an ordinary character
    \end{itemize}
  \end{itemize}
  \note{grep'}
\end{frame}

\begin{frame}{Regular expressions - examples}
  \begin{itemize}
  \item /abc/  matches ``abc''
  \item /[a-z]/ matches a lower case character
  \item /[a-z][A-Za-z0-9\_]+/ matches identifiers in many languages
  \item /\^[0-9]+\$/ matches a line containing only a single integer
  \item /(for)$\mid$(if)$\mid$(switch)$\mid$(while)/ matches some C language control keywords
  \end{itemize}
  \note{}
\end{frame}

\begin{frame}{Regular expressions - grep}
  \begin{itemize}
  \item grep (Global Regular Expression Print) is the standard search
    program
  \item grep -i PATTERN FILES - ignores case
  \item grep -l PATTERN FILES - prints just the FILES containing
    PATTERN
  \item grep -f RE\_FILE FILES - reads regular expressions, one per
    line, from RE\_FILE
  \end{itemize}
  \note{}
\end{frame}

\begin{frame}{}
%%  \includegraphics[width=10cm,scale=0.4]{images/man-k.png}
  \begin{itemize}
  \item 
  \end{itemize}
  \note{}
\end{frame}

\begin{frame}{}
%%  \includegraphics[width=10cm,scale=0.4]{images/man-k.png}
  \begin{itemize}
  \item 
  \end{itemize}
  \note{}
\end{frame}

\begin{frame}{}
%%  \includegraphics[width=10cm,scale=0.4]{images/man-k.png}
  \begin{itemize}
  \item 
  \end{itemize}
  \note{}
\end{frame}

\begin{frame}{}
%%  \includegraphics[width=10cm,scale=0.4]{images/man-k.png}
  \begin{itemize}
  \item 
  \end{itemize}
  \note{}
\end{frame}

\begin{frame}{}
%%  \includegraphics[width=10cm,scale=0.4]{images/man-k.png}
  \begin{itemize}
  \item 
  \end{itemize}
  \note{}
\end{frame}

\begin{frame}{Questions?}
  \note{}
\end{frame}

\end{document}
